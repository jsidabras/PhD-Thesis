\chapter[Introduction]{Introduction}



\section{Motivation of the Work.}
This work focuses on instrumentation developments, primarily manipulating the applied time-varying magnetic field within resonant probes, to increase both the quality and sensitivity of EPR spectroscopy for studying metallo-enzymes. 

First, a uniform field re-entrant TE$_{\text{01U}}$ is introduced in Chapter~3, which improves pulse performance with no loss in EPR signal intensity compared to conventional TE$_{011}$ cylindrical cavities by providing a 10~mm active region with strictly uniform magnetic field, an improved efficiency parameter along the sample, a low $Q$-value, and ability to be significantly over coupled. These improved characteristics optimize the re-entrant TE$_{\text{01U}}$ for frozen solution pulse EPR experiments. Such experiments are used to determine the principal values of the $g$-tensor or, using ELDOR-detected NMR (EDNMR), ESEEM, and/or HYSCORE spectroscopies, access hyperfine and quadrupole information of coupled nuclei. 

Second, a frequency-domain Fourier-transform EPR experiment has been performed with surface resonator arrays operating in the 420~GHz range and is described in Chapter~4. An analytical transmission-line model describing the system was formulated in order to understand the interactions between the complex magnetic susceptibility of the sample and the frequency dependant surface resonator array impedance. These analytical models are then used to further the understanding of the the interaction between the sample and resonant probe. Such distinctions become increasingly important as the resonant probe geometry volume approach the sample volume and in multi-wavelength probe designs found at higher frequencies. 

Finally, a self-resonant micro-helix  described in Chapter~5 has been designed and implemented resulting in a factor of 28 increase in absolute EPR signal intensity compared to commercially available cavities. Using this novel geometry, we address the problem of small volume sample sensitivity in EPR which becomes a limiting factor when studying valuable materials or crystals of various compounds such as RNA or proteins. The self-resonant micro helix has allowed, for the first time, single crystal \textit{g}-tensor orientation analysis and preliminary hyperfine analysis on the H-cluster of [FeFe]-hydrogenase in the H$_{ox}$ state on a protein single-crystal with dimensions of 0.3~$\times$~0.1~$\times$~0.1~mm$^3$. 

In total, this collection of work advances the state-of-the-art in EPR spectroscopy and, from such EPR advancements, further improves our understanding of the fundamental catalytic mechanism of [FeFe]-hydrogenase. 



{\renewcommand{\bibsection}{\clearpage\section*{\bibname}\markboth{\bibname}{\bibname}}
\renewcommand{\bibname}{CHAPTER 1. REFERENCES}
\bibliographystyle{elsarticle-num}
\bibliography{Kapitel/Ch5-References}
}