\chapter[Introduction]{Introduction}

Electron paramagnetic resonance (EPR) is a spectroscopic technique to probe the interactions of free electrons to the surrounding atomic environment. EPR has been employed to gain information on the structure and dynamics of proteins, radical reactions, electron transfer processes, and transition metal catalysis. 

\begin{itemize}
    \item Microwave magnetic field uniformity of cavity resonators 
    \item Thin-film samples at extreme operating frequencies
    \item 
\end{itemize}

The uniformity of the magnetic field in the cavity is typically cosinusoidal and , as such, means that the sample has different microwave incident field along the axis. This becomes more important as the number of pulses increase. Such as ESEEM, EDNMR, RIDME... etc... 



Three challenges are addressed in this work. Improvement of the uniformity of the microwave magnetic field is addressed based on the work of Sidabras, Mett, and Hyde. A uniform field re-entrant TE$_{\text{01U}}$ is introduced`, which improves pulse performance with no loss in EPR signal intensity compared to conventional TE$_{011}$ cylindrical cavities by providing a 10~mm active region with strictly uniform magnetic field, an improved efficiency parameter along the sample, a low $Q$-value, and ability to be significantly over coupled. These improved characteristics optimize the re-entrant TE$_{\text{01U}}$ for frozen solution pulse EPR experiments. Such experiments are used to determine the principal values of the $g$-tensor or, using ELDOR-detected NMR (EDNMR), ESEEM, and/or HYSCORE spectroscopies, access hyperfine and quadrupole information of coupled nuclei. 

Signal intensity for thin films of high spin paramagnetic materials. Study of zero-field splitting one needs to go to extreme frequencies. However, such frequencies typically need signifiant sample concentrations and material. A frequency-domain Fourier-transform EPR experiment has been performed with surface resonator arrays operating in the 420~GHz range. An analytical transmission-line model describing the system was formulated in order to understand the interactions between the complex magnetic susceptibility of the sample and the frequency dependant surface resonator array impedance. These analytical models are then used to further the understanding of the interaction between the sample and resonant probe. Such distinctions become increasingly important as the resonant probe geometry volume approach the sample volume and in multi-wavelength probe designs found at higher frequencies. 

Absolute sensitivity remains a challenge in EPR spectroscopy. Such volume limited samples include protein single crystal studies. Such studies advance the  understanding of the magnetic properties of a sample. From this knowledge one can piece together the mechanisms of the catalytic cycle. A self-resonant micro-helix has been designed and implemented resulting in a factor of 28 increase in absolute EPR signal intensity compared to commercially available cavities. Using this novel geometry, we address the problem of small volume sample sensitivity in EPR which becomes a limiting factor when studying valuable materials or small crystals of various compounds, such as proteins. 

The self-resonant micro helix has allowed, for the first time, single crystal $g$-tensor orientation analysis and preliminary hyperfine analysis on the active site of [FeFe]-hydrogenase in the H$_{ox}$ stable intermediate on a protein single-crystal with dimensions of 0.3~$\times$~0.1~$\times$~0.1~mm$^3$. 

In total, this collection of work advances the state-of-the-art in EPR spectroscopy and, from such EPR advancements, further improves our understanding of the fundamental catalytic mechanism of [FeFe]-hydrogenase. 


{\renewcommand{\bibsection}{\clearpage\section*{\bibname}\markboth{\bibname}{\bibname}}
\renewcommand{\bibname}{CHAPTER 1. REFERENCES}
\bibliographystyle{elsarticle-num}
\bibliography{Kapitel/Ch5-References}
}