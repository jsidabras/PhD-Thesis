\chapter[Introduction]{Introduction}
\setcitestyle{citesep={,\,\thechapter.}}
%   Introduction to the introduction: a short version (of only a few paragraphs) of the thesis' aims, research questions, contribution, objectives and findings.
%    State the overarching topic and aims of the thesis in more detail
%    Provide a brief review of the literature related to the topic (this will be very brief if you have a separate literature review chapter)
%    Define the terms and scope of the topic
%    Critically evaluate the current state of the literature on that topic and identify your gap
%    Outline why the research is important and the contribution that it makes
%    Outline your epistemological and ontological position
%    Clearly outline the research questions and problem(s) you seek to address
%    State the hypotheses (if you are using any)
%    Detail the most important concepts and variables 
%    Briefly describe your methodology
%    Discuss the main findings
%    Discuss the layout of the thesis

%    Does the first line of the introduction discuss the problem that your thesis is addressing and the contribution that it is making?
%    Does the introduction provide an overview of the thesis and end with a brief discussion on the content of each chapter?
%    Does the introduction make a case for the research?
%    Have the research questions/problems/hypotheses been clearly outlined (preferably early on)?
    
Electron paramagnetic resonance (EPR) is a spectroscopic technique to probe the interactions of free electrons to the surrounding atomic environment. EPR has been employed to gain information on the structure and dynamics of proteins, organic-based radicals, transition metals, radical reactions, electron transfer processes, and metallo-enzymes. 

As a magnetic resonance technique, EPR requires an incident microwave magnetic field perpendicular to a static magnetic field. 

This work focuses on the development of application-specific microwave resonators to solve the challenges of modern EPR spectroscopy. Specifically, three challenges are addressed in this work: 
\begin{enumerate}
\setlength\itemsep{-0.25em}
    \item Enhancement of the resonator microwave magnetic field homogeneity for pulse EPR.
    \item Enhancement of the EPR absorption profile thin films of high-spin paramagnetic materials in the THz-bandgap (100~GHz to 1~THz; 3.336-33.356~cm$^{-1}$).
    \item Increasing the absolute sensitivity of EPR for studying samples with volumes less than 27~nl. 
\end{enumerate}  

With the inclusion of arbitrary waveform generators to modern EPR spectrometers, it has become increasingly important to have control over the excitation profile of the sample. \cite{DOLL201418,WILI201826} Not only does the resonator bandwidth contribute to the EPR signal,\cite{Tschaggelar2017} but the profile of the magnetic field. This becomes more important as the number of pulses increase \cite{MILIKISYANTS200948,C7CP01488K,C6CP03067J,BREITGOFF2019106560} or when performing spin-spin distance measurements with radicals of different saturation profiles. \cite{C8CP01276H} Improvement of the uniformity of the microwave magnetic field is addressed based on the work of Sidabras, Mett, and Hyde. \cite{HydeUFRev2019}



To study samples with large zero-field splitting the 

In this context, The self-resonant micro helix has allowed, for the first time, single-crystal $g$-tensor orientation analysis and preliminary hyperfine analysis on the active site of [FeFe]-hydrogenase in the H$_{ox}$ stable intermediate on a protein single-crystal with dimensions of 0.3~$\times$~0.1~$\times$~0.1~mm$^3$. This advance in resonator development has opened up the doors to studying enzymatic activity in single crystals. by looking at stable intermediates with paramagnetic centers. S








This dissertation is compiled in the following way. In Chapter 2, the basic understanding of EPR is introduced. From this understanding the important characteristics of crystal symmetry as it relates to EPR is outlined along with electromagnetic finite-element simulations for resonator development. Each chapter is self-contained with an introduction, method, results, and conclusion section. 

A uniform field re-entrant TE$_{\text{01U}}$ is described, which improves pulse performance with no loss in EPR signal intensity compared to conventional TE$_{011}$ cylindrical cavities by providing a 10~mm active region with strictly uniform magnetic field, an improved efficiency parameter along the sample, a low $Q$-value, and ability to be significantly over coupled. These improved characteristics optimize the re-entrant TE$_{\text{01U}}$ for frozen solution pulse EPR experiments. Such experiments are used to determine the principal values of the $g$-tensor or, using ELDOR-detected NMR (EDNMR), ESEEM, and/or HYSCORE spectroscopies, access hyperfine and quadrupole information of coupled nuclei. 

Chapter 4 is concerned with thin-film samples in the THz-bandgap. Signal intensity for thin films of high spin paramagnetic materials. Study of zero-field splitting one needs to go to extreme frequencies. However, such frequencies typically need significant sample concentrations and material. A frequency-domain Fourier-transform EPR experiment has been performed with surface resonator arrays operating in the 420~GHz range. An analytical transmission-line model describing the system was formulated to understand the interactions between the complex magnetic susceptibility of the sample and the frequency-dependent surface resonator array impedance. These analytical models are then used to further the understanding of the interaction between the sample and the resonant probe. Such distinctions become increasingly important as the resonant probe geometry volume approaches the sample volume and in multi-wavelength probe designs found at higher frequencies. 

Absolute sensitivity remains a challenge in EPR spectroscopy. Chapter 5 focuses on increasing absolute sensitivity for sample volumes less than 85~nl. Such volume-limited samples include protein single-crystal studies. Single-crystal studies studies advance the understanding of the magnetic properties of a sample. From this knowledge, one can piece together the mechanisms of the catalytic cycle. A self-resonant micro-helix has been designed and implemented resulting in a factor of 30 increase in absolute EPR signal intensity compared to commercially available cavities. Using this novel geometry, we address the problem of small volume sample sensitivity in EPR which becomes a limiting factor when studying valuable materials or small crystals of various compounds, such as proteins. 

In chapter 6, the self-resonant micro-helix of Chapter 5 is employed to study [FeFe]-hydrogenase in the stable intermediate of H$_{ox}$. 







{\renewcommand{\bibsection}{\clearpage\section*{\bibname}\markboth{\bibname}{\bibname}}
\renewcommand{\bibname}{CHAPTER 1. REFERENCES}
\bibliographystyle{elsarticle-num}
\bibliography{Kapitel/Ch1-References}
}