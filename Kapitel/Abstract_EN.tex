\abstrakt{Abstract}
\vspace{-2em}
Electron paramagnetic resonance (EPR) is a spectroscopic technique to study the interaction between free electrons and the local molecular environment. Over the past 60 years, EPR instrumentation and methodology has been developed to study, for example, the structure and dynamics of proteins, chemical reactions of organic-based radicals, transition-metal chemistry, catalytic reactions, electron transfer processes, and metallo-enzymes. Because of these past developments, EPR has become a powerful tool for chemists and physicists alike. 

However, there remain several challenges in modern EPR that can be addressed by the development of application-specific microwave resonators. Three challenges are investigated: (i) improving the homogeneity of the magnetic flux density profile incident on a sample volume for applications to pulse EPR at Q-band frequencies (35~GHz); (ii) enhancing the sensitivity of EPR in the THz-bandgap (100~GHz to 1~THz frequency range; 3.34-33.36~cm$^{-1}$ energy range) to improve the detection of thin films with the use of a resonant meta-materials; and (iii) improving the absolute sensitivity at X-band frequencies (9.5~GHz) for the study of protein single-crystals with volumes less than 85~nl. First, the introduction of a uniform field re-entrant \cylTE{} cavity at Q-band frequencies provides a 10~mm region-of-interest with a microwave field uniformity of 98\%. The homogeneous field increases the microwave conversion factor by 60\%. This design implements a waveguide H-type T-junction coupler with inductive obstacles to improve the coupling efficiency. The resonator is tested with a standard sample and shown to significantly improve the excitation profile for pulse EPR. Second, an investigation of the interaction between a split-ring resonator meta-material surface and a protein sample is presented as a way to increase the EPR signal for sub-THz frequencies. Data is collected using Frequency-Domain Fourier-Transform THz-EPR in the energy range of 11-18~cm$^{-1}$. The interaction of the EPR signal with a meta-material resonating at 14~cm$^{-1}$ is modeled with a lumped-circuit transmission-line. It was found that both inductive and capacitive coupling is required to fully understand this complex system. From this analysis, a factor of 4 in the EPR signal is shown for an active sample height of 24~$\mu$m. Next, the absolute sensitivity at X-band frequencies has been increased up to a factor of 30 compared to commercial resonators by the implementation of a self-resonant micro-helix. This 0.4~mm inner diameter helix provides a resonator efficiency of 3.2~mT/W$^{1/2}$ corresponding to a $\pi/2$ pulse of 20~ns with an incident power of only 20~mW in a volume of 85~nl. This geometry is measured to have an absolute spin sensitivity of $64 \times 10^6$ Spins/G in 50 minutes of measurement time. Finally, the self-resonant micro-helix is used to obtain, for the first time, the angular dependence of the EPR signal from a protein single-crystal of [FeFe]-hydrogenase in the H$_{ox}$ state from {\em Clostridium pasteurianum} (CpI) with a volume of 3~nL. A signal-to-noise ratio of 290 was achieved for $4.25 \times 10^9$ spins in 8 minutes of measurement time. Full $g$-tensor analysis was successfully performed and an orientation of the principal axes is discussed. With the excellent signal-to-noise ratio, data was also collected on the same protein crystal using an ESEEM/HYSCORE pulse sequence.

In total, this work pushes the stat-of-the-art in EPR instrumentation allowing for new methodology development and broadening the applications available to chemists and physicists.