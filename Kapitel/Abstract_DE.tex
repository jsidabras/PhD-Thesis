\abstrakt{Zusammenfassung}
\vspace{-2em}
Elektronenparamagnetische Resonanz (EPR) ist eine biophysikalische spektroskopische Technik zur Untersuchung der Wechselwirkung zwischen freien Elektronen und deren lokaler molekularer Umgebung. In den letzten 60 Jahren wurden EPR-Instrumente und -Methoden entwickelt, um beispielsweise die Struktur und Dynamik von Proteinen, chemische Reaktionen organischer Radikale, Übergangsmetallchemie, katalytische Reaktionen, Elektronentransferprozesse und Metalloenzyme zu untersuchen. Aufgrund dieser Entwicklungen hat sich EPR zu einem leistungsstarken Werkzeug für Chemiker und Physiker entwickelt.

Jedoch verbleiben in der modernen EPR noch einige Herausforderungen, die durch die Entwicklung von anwendungsspezifischen Mikrowellenresonatoren adressiert werden können. In dieser Dissertation werden drei Herausforderungen untersucht: (i) Verbesserung der Homogenität des magnetischen Flussdichteprofils im Probenvolumen für Anwendungen in gepulster EPR bei Q-Band-Frequenzen (35~GHz); (ii) Erhöhen der Empfindlichkeit in der THz-Bandgap (100~GHz bis 1~THz Frequenzbereich; 3{,}34-33{,}36~cm$^{-1}$ Energiebereich) zur Verbesserung der Detektion von dünnen Schichten unter Verwendung eines resonanten Metamaterials; und (iii) Verbesserung der absoluten Empfindlichkeit bei X-Band-Frequenzen (9{,}5~GHz) für die Untersuchung von Protein-Einkristallen mit Volumina unter 30~nl. Erstens, die Einführung eines homogenen Feld ``re-entrant'' \cylTE{} Resonators für Q-Band-Frequenzen bietet einen 10~mm Bereich mit einer Mikrowellenfelduniformität von 98\%. Das homogene Feld erhöht den Mikrowellenkonversionsfaktor um 60\%. Dieses Design implementiert einen Wellenleiter H-Typ T-Übergangskoppler mit induktiven Hindernissen um die Kopplungseffizienz zu verbessern. Der Resonator wird mit einer Standardprobe getestet und zeigt eine signifikante Verbesserung des Anregungsprofiles für gepulste EPR. Zweitens wird eine Untersuchung der Wechselwirkung zwischen der Meta-Materialoberfläche eines Spaltringresonators und einer Proteinprobe als Möglichkeit zur Erhöhung des EPR-Signals für Sub-THz-Frequenzen vorgestellt. Die Daten werden mit Hilfe von Fourier-transformierter THz-EPR in der Frequenzdomäne im Energiebereich von 11-18~cm$^{-1}$ gesammelt.  Die Wechselwirkung zwischen dem EPR-Signal und einem Metamaterial mit einer Resonanz bei 14~cm$^{-1}$, wird mit einer lumped-circuit transmission-line modelliert. Es wurde festgestellt, dass für das vollständige Verständnis dieses komplexen Systems sowohl eine induktive als auch eine kapazitive Kopplung erforderlich ist. Bei einer aktiven Probenhöhe von 24~$\mu$m eine Verbesserung des EPR-Signals um einen Faktor 4 wird berichtet. Anschließend wurde die absolute Empfindlichkeit bei X-Band-Frequenzen gegenüber kommerziellen Resonatoren durch die Implementierung einer selbstresonanten Mikrohelix um den Faktor 30 erhöht. Diese Helix mit einem 0{,}4~mm Innendurchmesser bietet einen Resonatorwirkungsgrad von 3{,}2~mT/W$^{1/2}$, was einem 20~ns $\pi/2$-Puls bei einer Leistung von nur 20~mW in einem Volumen von 85~nl entspricht. %Mit dieser Geometrie kann eine absolute Spin-Empfindlichkeit von $64\times10^6$ Spins/G in 50 Minuten Messzeit erreicht werden. 
Schließlich kann mit dieser selbstresonanten Mikrohelix erstmals die Winkelabhängig\-/keit des EPR-Signals eines Proteineinkristalls der [FeFe]-Hydrogenase im H$_{ox}$ Zustand aus \textit{Clostridium pasteurianum} (CpI) mit einem Volumen von 3~nL erhalten werden. 
%Ein Signal-Rausch-Verhältnis von 290 wurde für $4{,}25\times10^9$ Spins in 8 Minuten Messzeit erreicht. 
Eine vollständige $g$-Tensoranalyse wurde erfolgreich durchgeführt und die Orientierung der Hauptachsen diskutiert. Mit dem ausgezeichneten Signal-Rausch-Verhältnis wurden auch Daten mit ESEEM/HYSCORE-Pulssequenzen auf demselben Proteinkristall gesammelt.

Insgesamt treibt diese Arbeit den Stand der Technik in der EPR-Instrumentierung voran, was eine neue Methodenentwicklung und eine Erweiterung der Anwendungen für Chemiker und Physiker ermöglicht.

