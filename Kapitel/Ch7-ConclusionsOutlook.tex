\chapter[Summary and Future Work]{Summary and Future Work}
\setcitestyle{citesep={,\,\thechapter.}}

In this body of work, three challenges of modern EPR were addressed by the development of application-specific microwave resonant structures. These efforts span the frequency range from 9.5 to 420~GHz.

In Chapter 3, the introduction of the uniform field re-entrant TE$_{\text{01U}}$ resonator at Q-band frequencies (35~GHz) improves EPR sensitivity by providing more homogeneous $\pi/2$ and $\pi$ pulses incident on the sample. This has an advantage in pulse experiments with complex sequences used for excitation of the spin system. 

For example, HYSCORE data at Q-band will benefit from uniform magnetic field excitation by providing a mixing pulse which will maximize the nuclear transition mixing and remove on-diagonal signals. \cite{Doorslaer2007,Harmer2009} Such signals typically dominate the spectrum and make interpretation difficult. For EDNMR, a uniform magnetic field excitation translates into a reduction of the width of the central peak allowing for the measurement of modulation frequencies typically hidden by this feature. \cite{NicholasCox2013} Finally, the increase in resonator bandwidth, from the lower Q-value and implementation of a more efficient waveguide junction, allows the spectroscopist to utilize state-of-the-art arbitrary waveform generators for further signal improvement and method development. \cite{DOLL201327,dSegawa2015,SPINDLER201730,WILI201826,PRISNER201998}

In Chapter~4, the experiments in the THz-bandgap with hemin sample and meta-materials made from an array of split-ring resonators at 420~GHz serve as an interesting example of the two oscillator problem in physics. Using analytical lumped-circuit transmission-line theory, the EPR response was fully reproduced. The use of split-ring resonators creates a cross-sectional area that is not limited by the wavelength of the operating frequency. The split-ring resonators maintain an active depth proportional to the individual geometry providing a suitable volume for thin-film samples. An EPR signal enhancement is measured at the operating frequency of the split-ring resonators. With the new understanding outlined in this work, it may be possible to design a series of multi-frequency high-field EPR experiments with meta-material resonant discs at each frequency. This will allow spectroscopists to probe the zero-field splitting characteristics of high-spin samples in thin films or limited samples.

The introduction of the self-resonant micro-helix at X-band (9.5~GHz) in Chapter~5 provides a resonator efficiency of 3.2~mT/W$^{1/2}$ corresponding to a $\pi/2$ pulse of 20~ns with an incident power of only 20~mW. The self-resonant micro-helix exhibits an absolute sensitivity increase up to a factor of approximately 30 in the EPR signal compared to commercial resonators. For samples that saturate readily, the micro-helix provides a factor of 6 in absolute EPR sensitivity improvement compared to commercial resonators. This translates into a factor of 36 in measurement time for equivalent sample volumes. Additionally, the high resonator efficiency and low $Q_0$-value (bandwidth of 90~MHz critically coupled) provides an opportunity to measure FID induced EPR on systems without costly high-power amplifiers, further extending the usefulness of pulse EPR spectroscopy.

Finally, the implementation of the self-resonant micro-helix has allowed, for the first time, the collection of EPR data from a 0.3 $\times$ 0.1 $\times$ 0.1~mm$^3$ single crystal of [FeFe]-hydrogenase in the H$_{ox}$ state from {\em Clostridium pasteurianum} (CpI). Full $g$-tensor analysis was successfully performed and a proposed orientation of the principal axes is discussed. With the excellent signal-to-noise ratio, data was also collected on the same protein crystal using an ESEEM/HYSCORE pulse sequence. These data show an angle dependant $^{14}$N hyperfine- and quadrupole-tensor originating from either the cyanide-ligand of the distal iron or the ADT-ligand.  To our knowledge, the ESEEM/HYSCORE spectra collected herein are the first published results from a protein single-crystal with a volume of 3~nL at X-band. 

\subsection{Future Work}

In total, this collection of work advances the state-of-the-art of EPR spectroscopy. However, within these challenges, the most exciting is the application of single-crystal EPR to metalloenzyme research.

Further advancements to the assembly described in Chapter 5 is the implementation of onboard cryogenic low-noise amplifiers. Since most metalloenzyme EPR is performed in the 5-20~K temperature range, the use of cryogenic low-noise amplifiers is straight-forward. It was not pursued in this work since the commercial bridge would have to be modified to accommodate a transmission probehead. However, at least a factor of three is possible. \cite{NARKOWICZ201379} 

The implementation of single-crystal EPR to metalloenzymes allows for the study of many complex reaction centers. \cite{holm2014introduction} Not only does this research push the boundaries of detection limits, but it also provides experimental data for quantum chemical calculations that lack accurateness for open-shell molecules. Studies of metalloenzymes, such as [FeFe]-hydrogenase, provide much-needed data to study the interactions of the active site to the first and second coordination spheres. Ultimately, this work would lead to better synthetic enzymes as the mechanisms for electron transfer and magnetic-molecular interactions are better understood. 




{\renewcommand{\bibsection}{\clearpage\section*{\bibname}\markboth{\bibname}{\bibname}}
\renewcommand{\bibname}{REFERENCES}
\bibliographystyle{elsarticle-num}
\bibliography{Kapitel/Ch5-References}
}