\chapter[Summary and Future Work]{Summary and Future Work}
\setcitestyle{citesep={,\,\thechapter.}}

In this body of work, three challenges of modern EPR were addressed by the development of application-specific microwave resonant structures. These efforts span the frequency range from 9.5 to 420~GHz.

In Chapter 3, the introduction of the uniform field re-entrant TE$_{\text{01U}}$ resonator at Q-band frequencies (35~GHz) improved EPR sensitivity by providing more homogeneous magnetic flux density over the sample volume during $\pi/2$ and $\pi$ pulses. This has an advantage in pulse experiments as the sequences used for excitation of the spin system become more complex. 

For example, HYSCORE data at Q-band will benefit from uniform magnetic flux density excitation by providing a mixing pulse that maximizes the nuclear transition mixing and reduce on-diagonal signals. \cite{Doorslaer2007,Harmer2009} Such signals dominate the spectrum and make interpretation difficult. For EDNMR, a uniform magnetic field excitation translates into a reduction of the width of the central peak allowing for the measurement of modulation frequencies typically hidden by this feature. \cite{NicholasCox2013} For ESEEM and pulsed electron electron double resonance (PELDOR), the modulation depth improves due to the full excitation of the spins. Finally, the increase in resonator bandwidth, from the lower Q-value and implementation of a more efficient waveguide junction, allows the spectroscopist to utilize state-of-the-art arbitrary waveform generators for further signal improvement and method development. \cite{DOLL201327,dSegawa2015,SPINDLER201730,WILI201826,PRISNER201998}

In Chapter~4, the experiments in the THz-bandgap with hemin sample and meta-materials made from an array of split-ring resonators at 420~GHz served as an interesting example of the two oscillator problem in physics. Using analytical lumped-circuit transmission-line theory, the EPR response was fully reproduced. The use of split-ring resonators created a cross-sectional area that was not limited by the wavelength of the operating frequency. The split-ring resonators maintain an active depth proportional to the individual geometry providing a suitable volume for thin-film samples. An EPR signal enhancement of 4 was measured at the operating frequency of the split-ring resonators compared to the system without the meta-materials coupled. With the new understanding outlined in this work, it may be possible to design a series of multi-frequency high-field EPR experiments with meta-material resonant discs at each frequency. This will allow spectroscopists to probe the zero-field splitting characteristics of high-spin samples in thin-film or limited samples.

The self-resonant micro-helix at X-band (9.5~GHz) in Chapter~5 provided a resonator efficiency of 3.2~mT/W$^{1/2}$ corresponding to a $\pi/2$ pulse of 20~ns with an incident power of only 20~mW. This self-resonant micro-helix exhibited an absolute sensitivity increase up to a factor of approximately 30 in the EPR signal compared to commercial resonators. For samples that saturate readily, the micro-helix provided a factor of 6 in absolute EPR sensitivity. This translates into a factor of 36 in measurement time for equivalent sample volumes. Additionally, the high resonator efficiency and low $Q_0$-value (bandwidth of 90~MHz critically coupled) provides an opportunity to measure FID induced EPR on systems without costly high-power amplifiers, further extending the usefulness of pulse EPR spectroscopy.

Finally, the implementation of the self-resonant micro-helix has allowed, for the first time, the collection of EPR data from a single crystal of [FeFe]-hydrogenase in the H$_{ox}$ state from {\em Clostridium pasteurianum} (CpI). The crystal dimensions were limited by by both the duration of crystal growth time and the general crystal growth size of 0.3 $\times$ 0.1 $\times$ 0.1~mm$^3$. Full $g$-tensor analysis was successfully performed and an orientation of the principal axes was discussed. With the excellent signal-to-noise ratio, data was also collected on the same protein crystal using an ESEEM/HYSCORE pulse sequence. These data show an angle dependant $^{14}$N hyperfine- and quadrupole-tensor originating from the cyanide-ligand of the distal iron and the ADT-ligand. To our knowledge, the ESEEM/HYSCORE spectra collected herein were the first published results from a protein single-crystal with a volume of 3~nL at X-band. 

\subsection*{Future Work}

In total, this collection of work advances the state-of-the-art of EPR spectroscopy. However, within these challenges, the most exciting is the application of single-crystal EPR to metallo-enzyme research. 

Further advancements to the assembly described in Chapter 5 include the implementation of onboard cryogenic low-noise amplifiers. Since most metallo-enzyme EPR is performed in the 5-20~K temperature range, the use of cryogenic low-noise amplifiers is straight-forward. It was not pursued in this work since the commercial bridge would have to be modified to accommodate a transmission probehead. However, at least a factor of three increase in EPR signal-to-noise ratio compared to the work presented here is possible. \cite{NARKOWICZ201379} 

The implementation of single-crystal EPR to metallo-enzymes allows for the study of many complex reaction centers. \cite{holm2014introduction} Not only does this research push the boundaries of detection limits, but it also provides experimental data for quantum chemical calculations that lack accurateness for open-shell molecules. Single-crystal studies of metallo-enzymes, such as [FeFe]-hydrogenase, provide the much-needed data to study the interactions of the active site to the first and second coordination spheres. 

Furthermore, other potentially interesting proteins, such as CODH \cite{C5CS00182J}, MMO \cite{Hoffman2014rev}, Rieske \cite{FERRARO2005175} and other Fe-S cluster containing proteins \cite{FeSClustersReview} are rarely studied in single crystals. Yet, in doing so, the electronic structure and enzymatic function of these systems could be extensively studied building on the synergy of X-ray crystallography diffraction for structural information and EPR for structural, function, and dynamic information of the local molecular interactions of an enzyme.  


{\renewcommand{\bibsection}{\clearpage\section*{\bibname}\markboth{\bibname}{\bibname}}
\renewcommand{\bibname}{CHAPTER 7. REFERENCES}
\bibliographystyle{elsarticle-num}
\bibliography{Kapitel/Ch5-References}
}