\chapter[Finite-element Modeling Signal Calculations]{Finite-element Modeling Signal Calculations}
\chaptermark{\textit{CalculateSignal-Water.cls}}
 	\definecolor{light-gray}{gray}{0.90}
\lstset{language=Perl,%
    %basicstyle=\color{red},
    breaklines=true,%
    morekeywords={matlab2tikz},
    keywordstyle=\color{blue},%
    morekeywords=[2]{1}, keywordstyle=[2]{\color{black}},
    identifierstyle=\color{black},%
    stringstyle=\color{mylilas},
    commentstyle=\color{mygreen},%
    showstringspaces=false,%without this there will be a symbol in the places where there is a space
    numbers=left,%
    tabsize=2,
    frame=single,
    backgroundcolor = \color{light-gray},
    rulecolor=\color{black},
    numberstyle={\tiny \color{black}},% size of the numbers
    numbersep=9pt, % this defines how far the numbers are from the text
    emph=[1]{for,end,break},emphstyle=[1]\color{red}, %some words to emphasise
    %emph=[2]{word1,word2}, emphstyle=[2]{style},    
}

\section*{Signal Calculations for Ansys HFSS Assuming an Aqueous Sample}

The following lines of code can be placed in a file called ``CalculateSignal-Water.cls'' and placed in the \textit{PersonalLibs} folder from the Ansys installation. The \textit{cls} file should be loaded in the \textit{\textbf{Fields Calculator}} twice, in order to load all expressions.\footnote{This seems to be a dependency bug.} 

All codes can be found at https://github.com/jsidabras/HFSSTutorial/.


\subsection*{Constants}
The following constants are defined. 
\begin{itemize}
    \item \textbf{ImDieHold}: The imaginary dielectric constant ($\epsilon''$) of the dielectric that holds the resonator. In the case of the PMR, this is the sapphire. In the case of the micro-Helix, this is the PTFE.
    \item \textbf{ImDieSam}: The imaginary dielectric constant ($\epsilon''$) of the sample. In the case of water, the dielectric constant at X-band (9.5~GHz) is $\epsilon_r = 63 - i 26.46$. In the case of ICE, the dielectric constant at X-band is $\epsilon_r = 3.2 - i 0.00128$
    \item \textbf{Frq}: The real part of the solved frequency.
\end{itemize}{}
\lstinputlisting[linerange=1-16]{Kapitel/Appendix/Signal-water.clc}
\newpage

\subsection*{Power losses}
The following power losses are being defined
\begin{itemize}
    \item \textbf{Pls}: Losses associated with the sample. Uses \textit{ImDieSam} as $\epsilon''$.
    \item \textbf{Plw}: Requires the user to select all faces of objects that are metal and create \textit{FaceList1} (Modeler$\xrightarrow{}$List$\xrightarrow{}$Create$\xrightarrow{}$Face List).
    \item \textbf{Ple}: Losses associated with the dielectric holder. Uses \textit{ImDieHold} as $\epsilon''$.
\end{itemize}{}
\lstinputlisting[linerange=59-123]{Kapitel/Appendix/Signal-water.clc}
\newpage

\subsection*{Magneic Fields}
\lstinputlisting[linerange=17-58]{Kapitel/Appendix/Signal-water.clc}
\newpage


\subsection*{Signal}
\lstinputlisting[linerange=124-164]{Kapitel/Appendix/Signal-water.clc}
\newpage

\subsection*{Resonator Efficiency}
\lstinputlisting[linerange=165-203]{Kapitel/Appendix/Signal-water.clc}